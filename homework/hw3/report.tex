\documentclass{article}

\input{preamble.tex}

\makeatletter
\newcommand{\skipitems}[1]{%
  \addtocounter{\@enumctr}{#1}%
}
\makeatother

\begin{document}
    \homework{COM-514: Mathematical Foundations of Signal Processing}{Fall 2019}{3}{20\textsuperscript{th} December 2019}{Oriol Barbany Mayor}
    
    \problem{Maximum SNR Beamformer}
    \problem{Splines (I)}
    \begin{enumerate}[label=(\roman*)]
        \item To find the values $a_1,a_2$ it's enough to impose the boundary conditions
        \begin{align}
            s_1(1)=a_1 = f(1)=\frac{1}{1} \Longrightarrow a_1 = 1 \qquad ; \qquad s_2(4)=a_2=f(4)=\frac{1}{4} \Longrightarrow a_2 = \frac{1}{4}
        \end{align}
        \item The rows of the matrix correspond to the following constraints:
        \begin{enumerate}[label=\arabic*.]
            \item $s_1'(1)=f'(1)=\frac{-1}{1^2}$
            \item $s_1(4)=f(4)=\frac{1}{4}$
            \item $s_1'(4)=s_1'(4)$
            \item $s_1''(4)=s_1''(4)$
            \item $s_2'(4)=f'(4)=\frac{-1}{4^2}$
            \item $s_2(5)=f(5)=\frac{1}{5}$
        \end{enumerate}
        \begin{align}
            \begin{bmatrix}
            1 & 0 & 0 & 0 & 0 & 0 \\
            (4-1) & (4-1)^2 & (4-1)^3 & 0 & 0 & 0\\
             1 & 2(4-1) & 3(4-1)^2 & -1 & -2(4-1) & -3(4-1)^2 \\
            0 & 2 & 3(4-1) & 0 & -2 & -3(4-1) \\
            0 & 0 & 0 & 1 & 0 & 0 \\
            0 & 0 & 0 & (5-4) & (5-4)^2 & (5-4)^3 \\
            \end{bmatrix}
            \begin{bmatrix}
                b_1 \\
                c_1 \\
                d_1 \\
                b_2 \\
                c_2 \\
                d_2
            \end{bmatrix}
            =
            \begin{bmatrix}
                -1 \\
                \frac{1}{4} - 1 \\
                0 \\
                0 \\
                -\frac{1}{16} \\
                \frac{1}{5}-\frac{1}{4} \\
            \end{bmatrix}
        \end{align}
        \item Let $d(x):=f(x)-s(x)$. Then, using integration by parts in each subinterval,
        \begin{align}
            \int_a^b s''(x)d''(x)dx&:=\int_a^b s''(x)(f''(x) - s''(x))dx\\
            &=\sum_{i=1}^{n-1}\int_{x_i}^{x_{i+1}} s''(x)(f''(x) - s''(x))dx \\
            &=\sum_{i=1}^{n-1} s''(x)(f'(x)-s'(x))\bigg\rvert_{x_i}^{x_{i+1}} - \int_{x_i}^{x_{i+1}}  s'''(x)(f'(x) - s'(x))dx
            \label{eq:subinterval}
        \end{align}
        
        The spline is chosen so that $f(x_i)=s(x_i) \ \forall i\in [1,2,\dots, n]$. Moreover, given that $s$ is a cubic Spline, the function $s'''$ is a constant, namely $c\in \mathbb{C}$. Hence, 
        \begin{align}
            \int_{x_i}^{x_{i+1}}  s'''(x)(f'(x) - s'(x))dx &= c \int_{x_i}^{x_{i+1}}(f'(x) - s'(x))dx \\
            &= c[f(x_i)-f(x_{i+1})-(s(x_i)-s(x_{i+1}))]=0
            \label{eq:const}
        \end{align}
        
        Mixing \eqref{eq:subinterval} and \eqref{eq:const} and telescoping,
        \begin{align}
            \int_a^b s''(x)d''(x)dx&:=\sum_{i=1}^{n-1} s''(x)(f'(x)-s'(x))\bigg\rvert_{x_i}^{x_{i+1}} \\
            &= \sum_{i=1}^{n-1} s''(x_{i+1})(f'(x_{i+1})-s'(x_{i+1})) - s''(x_{i})(f'(x_{i})-s'(x_i)) \\
            &=s''(b)(f'(b)-s'(b)) - s''(a)(f'(a)-s'(a)) =0
            \label{eq:0}
        \end{align}
        where the last equality follows by the clamped boundary conditions.
        
        \item Using \eqref{eq:0}, we have that
        \begin{align}
            \int_a^b s''(x)(s''(x) - f''(x)) dx = 0 \Longleftrightarrow \int_a^b [s''(x)]^2dx = \int_a^b s''(x) f''(x) dx
            \label{eq:equality}
        \end{align}
        
        Given the latter, we can write
        \begin{align}
           \int_{a}^b [f''(x)-s''(x)]^2 dx &=  \int_{a}^b [f''(x)]^2 dx -  2\int_{a}^b f''(x)s''(x)dx +  \int_{a}^b [s''(x)]^2 dx \\
           &\refEQ{eq:equality} \int_{a}^b [f''(x)]^2 dx -  2\int_{a}^b [s''(x)]^2 dx +  \int_{a}^b [s''(x)]^2 dx \\
           &=\int_{a}^b [f''(x)]^2 dx - \int_{a}^b [s''(x)]^2 dx \geq 0
        \end{align}
        where the last inequality follows since we are integrating a non-negative function. The proof concludes by rearranging these terms:
        \begin{align}
          \int_{a}^b [f''(x)]^2 dx \geq \int_{a}^b [s''(x)]^2 dx
        \end{align}
    \end{enumerate}
    \problem{Splines (II)}
    \begin{enumerate}
        \item The minimum value of $K$ would be 1, since the signal $x(t)$ is piece-wise linear.
        \item Let $x(t)=\sum_{k \in \mathbb{Z}}\alpha_k \beta_+^{(1)}(t-k)$. Given that the B-spline of degree 1 is defined as a triangle of height 1 from 0 to 1, and the signal $x(t)$ is piece-wise linear, it's easy to find the values of $\alpha_k$ graphically.
        \begin{align}
            \alpha = \begin{bmatrix}
                \cdots &
                0 &
                \fbox{1} &
                1 &
                3 &
                -1 &
                0 &
                \cdots
            \end{bmatrix}^T
            \label{eq:alpha}
        \end{align}
        \item Using Fubini's theorem,
        \begin{align}
            \int_{-\infty}^t x(\tau) d\tau &= \int_{-\infty}^t\sum_{k \in \mathbb{Z}}\alpha_k \beta_+^{(1)}(\tau-k) d\tau=\sum_{k \in \mathbb{Z}}\alpha_k \int_{-\infty}^t \beta_+^{(1)}(\tau-k) d\tau =\sum_{k \in \mathbb{Z}}\alpha_k \int_{-\infty}^{t-k} \beta_+^{(1)}(s) ds 
        \end{align}
        
        Given that the causal elementary B-spline of degree $K$ is defined as $\beta_+^{(K)}:=\beta_+^{(K-1)} \ast \beta_+^{(0)}$,
        \begin{align}
        \int_{-\infty}^t x(\tau) d\tau  &= \sum_{k \in \mathbb{Z}}\alpha_k \int_{-\infty}^{t-k} \beta_+^{(1)}(s) ds= \sum_{k \in \mathbb{Z}}\alpha_k \sum_{m=0}^{\infty}\int_{t-k-m-1}^{t-k-m} \beta_+^{(1)}(s) ds \\
        &=: \sum_{k \in \mathbb{Z}}\alpha_k\sum_{m=0}^{\infty}\int_{-\infty}^{\infty} \beta_+^{(1)}(s) \beta_+^{(0)}(t-k-m-s) ds \\
        &= \sum_{k \in \mathbb{Z}}\alpha_k\sum_{m=0}^{\infty} (\beta_+^{(1)} \ast  \beta_+^{(0)})((t-k-m)) 
        \\&:= \sum_{k \in \mathbb{Z}}\alpha_k\sum_{m=0}^{\infty} \beta_+^{(2)}(t-k-m) \\
        &=\sum_{k \in \mathbb{Z}}\alpha_k \sum_{n=k}^{\infty} \beta_+^{(2)}(t-n)=\sum_{n \in \mathbb{Z}} \left[\sum_{k=-\infty}^{n}\alpha_k \right] \beta_+^{(2)}(t-n) \\
        &:= \sum_{n \in \mathbb{Z}} b_n \beta_+^{(2)}(t-n)
        \end{align}
        
        So, using the latter and \eqref{eq:alpha}, we have that
        \begin{align}
            \alpha = \begin{bmatrix}
                \cdots &
                0 &
                \fbox{1} &
                2 &
                5 &
                4 &
                4 &
                \cdots
            \end{bmatrix}^T
        \end{align}
        
    \end{enumerate}
\end{document}

