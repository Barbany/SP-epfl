\documentclass{article}

\input{preamble.tex}

\makeatletter
\newcommand{\skipitems}[1]{%
  \addtocounter{\@enumctr}{#1}%
}
\makeatother

\begin{document}
    \homework{COM-514: Mathematical Foundations of Signal Processing}{Fall 2019}{2}{29\textsuperscript{th} November 2019}{Oriol Barbany Mayor}
    
    \problem{Quick Review of Chapter 2}
    \begin{enumerate}[label=(\roman*)]
        \item
        \begin{itemize}
            \item $A(x)=x * h_A, \quad x, A_h\in \ell^2$. It's easy to check that $A$ is a linear operator. Let $y\in \ell^2$. 
            \begin{align}
                (A(\alpha x + \beta y))_n&=((\alpha x + \beta y) * h_A)_n = \sum_{k\in \mathbb{Z}} (\alpha x[k] + \beta y[k]) h_A[n-k]\\
                &=\alpha \sum_{k\in \mathbb{Z}} x[k]h_A[n-k] + \beta \sum_{k\in \mathbb{Z}} y[k] h_A[n-k] = \alpha (A(x))_n + \beta (A(y))_n
            \end{align}
            
            Moreover, $A$ is also shift invariant so it's LSI. To see this, let $x'[n]:=x[n-n_0]$
            \begin{align}
                (A(x'))_n= x' * h_A := \sum_{k\in \mathbb{Z}} x'[n-k] h_A[k] := \sum_{k\in \mathbb{Z}} x[n-k-n_0] h_A[k] = (A(x))_{n-n_0}
            \end{align}
            \item $B(x)(t)=x(t)+\text{sinc}(t),\quad x\in \mathcal{L}^2(\mathbb{R})$ is clearly not shift invariant since for $x'(t):=x(t-t_0)$
            \begin{align}
                B(x')(t)=x'(t)+\text{sinc}(t):=x(t-t_0)+\text{sinc}(t)\neq B(x)(t-t_0)
            \end{align}
            hence $B$ is not LSI.
            
            \item $C(x)(t)=x(2t),\quad x \in \mathcal{L}^2(\mathbb{R})$. Let $y\in \mathcal{L}^2(\mathbb{R})$. $C$ is a linear operator since
            \begin{align}
                C(\alpha x + \beta y)(t) = (\alpha x + \beta y)(2t) = \alpha x(2t) + \beta y(2t)
            \end{align}
            
            To check whether it's also shift invariant, let $x'(t):=x(t-t_0)$
            \begin{align}
                C(x')(t) = x'(2t) := x(2(t-t_0))=C(x)(t-t_0)
            \end{align}
            hence $C$ is LSI.
            
            \item $D(x)=\frac{dx}{dt},\quad x\in \mathcal{C}^\infty$. It's well-known that the derivative is linear, but for completeness, let $y\in \mathcal{C}^\infty$
            \begin{align}
                D(\alpha x + \beta y) = \frac{d}{dt}(\alpha x + \beta y) = \alpha \frac{dx}{dt} + \beta \frac{dy}{dt}
            \end{align}
            so indeed linearity holds for $D$. To check shift invariance, let $x'(t):=x(t-t_0)$ and by chain rule we have that
            \begin{align}
                D(x')(t)=\frac{d}{dt}x'(t) := \frac{d}{dt}x(t-t_0) =  D(x)(t-t_0) \left[\frac{d}{dt}(t-t_0)\right] =  D(x)(t-t_0)
            \end{align}
            so $D$ is also LSI.
        \end{itemize}
        
        \skipitems{1}
        \item By definition of the adjoint operator, we have that
        \begin{align}
            \lin{C(x)(t),y(t)}_{\mathcal{L}^2(\mathbb{R})} &= \int_{-\infty}^{\infty} C(x)(t)y^*(t) dt = \int_{-\infty}^{\infty} x(2t) y^*(t) dt \\
            &= \lin{x(t),C^*(y)(t)}_{\mathcal{L}^2(\mathbb{R})} = \int_{-\infty}^{\infty} x(t) (C^*(y)(t))^* dt
        \end{align}
        so we can see that by letting $t':=2t$,
        \begin{align}
            \int_{-\infty}^{\infty} x(2t) y^*(t) dt = \int_{-\infty}^{\infty} x(t') y^*\left(\frac{t'}{2}\right) dt'= \int_{-\infty}^{\infty} x(t) (C^*(y)(t))^* dt
        \end{align}
        and hence $C^*(y)(t)=y\left(\frac{t}{2}\right)$.
        \item Let $x,y\in \mathcal{H}$ and let $c\geq 0$ be such that $x(t)=0$ for $|t|\geq c$ and $d\geq 0$ such that $y(t)=0$ for $|t|\geq d$. Now by definition of the adjoint operator,
        \begin{align}
            \lin{D(x)(t),y(t)}_{\mathcal{H}} &= \int_{\mathcal{H}} D(x)(t)y^*(t) dt = \int_{-d }^{d} \frac{dx}{dt}(t) y^*(t) dt \\
            &= \lin{x(t),D^*(y)(t)}_{\mathcal{H}} = \int_{\mathcal{H}} x(t) (D^*(y)(t))^* dt
        \end{align}
        
        Using integration by parts, we get that
        \begin{align}
            \int_{-d }^{d} \frac{dx}{dt}(t) y^*(t) dt &=  x(t) y^*(t) \big\rvert_{-\min(c,d)}^{\min(c,d)} -  \int_{\mathcal{H}} x(t) \frac{dy^*}{dt}(t) dt \\
            &= -  \int_{\mathcal{H}} x(t) \frac{dy^*}{dt}(t) dt
        \end{align}
        where the last equality follows since $\mathcal{H}\subset \mathcal{L}^2(\mathbb{R})$, so $y^* = y$, and $x(\pm \min(c,d))y(\pm \min(c,d))=0$ by the finite support property of $\mathcal{H}$ and the definition of $c,d$. So $D^*(y)(t)=-\frac{dy}{dt}(t)$.
        
    \end{enumerate}
    
    \problem{LCMV and GSC Derivation}
    \begin{enumerate}[label=(\roman*)]
        \item If $\xx=A^* \yy$ and the solution must satisfy $A\xx= \bb$, then it's clear that $AA^* \yy = \bb$.
        \item
        \item 
        \item
    \end{enumerate}
\end{document}

