\documentclass{article}

\usepackage[utf8]{inputenc}

\usepackage[top=2cm, bottom=3.5cm, left=2.5cm, right=2.5cm]{geometry}

\usepackage{enumitem}
\usepackage{booktabs}
\usepackage[T1]{fontenc}
\usepackage{float}

\usepackage{graphicx}
\usepackage{epstopdf}

%
% The following macro is used to generate the header.
% Arguments are
% Course, Semester+Year, Homework number, Due Date, Student
%
\newcommand{\homework}[5]{
   \thispagestyle{plain}
   \newpage
   \noindent
   \begin{center}
   \framebox{
      \vbox{\vspace{2mm}
    \hbox to 6.28in { {\bf #1 \hfill #2} }
       \vspace{6mm}
       \hbox to 6.28in { {\Large \hfill Homework \##3 - Due date: #4\hfill} }
       \vspace{4mm}
       \hbox to 6.28in { {\hfill Student: #5} }
      \vspace{2mm}}
   }
   \end{center}
}

% Style of greek letters
\renewcommand{\phi}{\varphi}
\renewcommand{\epsilon}{\varepsilon}

\usepackage{titlesec}
\usepackage{sectsty}
\allsectionsfont{\mdseries\scshape}
\titlelabel{{\mdseries\scshape Problem }\thesection\ -\ }
\newcommand{\problem}{\section}

% Macros EPFL-MLO
\usepackage{amssymb,amsmath,amsthm,dsfont}

% Indicator function
\usepackage{bbm}
\providecommand{\ind}[1]{\mathbbm{1}_{\{#1\}}}

\providecommand{\lin}[1]{\ensuremath{\left\langle #1 \right\rangle}}
\providecommand{\abs}[1]{\ensuremath{\left\lvert#1\right\rvert}}
\providecommand{\norm}[1]{\ensuremath{\left\lVert#1\right\rVert}}

\providecommand{\refLE}[1]{\ensuremath{\stackrel{(\ref{#1})}{\leq}}}
\providecommand{\refEQ}[1]{\ensuremath{\stackrel{(\ref{#1})}{=}}}
\providecommand{\refGE}[1]{\ensuremath{\stackrel{(\ref{#1})}{\geq}}}
\providecommand{\refID}[1]{\ensuremath{\stackrel{(\ref{#1})}{\equiv}}}

  % basic sets
  \providecommand{\R}{\mathbb{R}} % Reals
  \providecommand{\N}{\mathbb{N}} % Naturals
  
  % random variables
  \DeclareMathOperator{\E}{{\mathbb E}}
  %\providecommand{\E}[1]{{\mathbb E}\left.#1\right. }     %expectation
  \providecommand{\Eb}[1]{\ensuremath{\E \left[#1\right] }} %expectation, with brackets
  \providecommand{\EE}[2]{\E_{#1} \! #2 }      %expectation  
  \providecommand{\EEb}[2]{\ensuremath{\E_{#1}\!\! \left[#2\right] }} %expectation,  with brackets
  \providecommand{\prob}[1]{\ensuremath{{\rm Pr}\left[#1\right] } }
  \providecommand{\Prob}[2]{\ensuremath{{\rm Pr}_{#1}\left[#2\right] } }
  \providecommand{\P}[1]{\ensuremath{{\rm Pr}\left.#1\right. }}
  \providecommand{\Pb}[1]{\ensuremath{{\rm Pr}\left[#1\right] }}
  \providecommand{\PP}[2]{\ensuremath{{\rm Pr}_{#1}\left[#2\right] }}
  \providecommand{\PPb}[2]{\ensuremath{{\rm Pr}_{#1}\left[#2\right] }}
  
  \newcommand\independent{\protect\mathpalette{\protect\independenT}{\perp}}
  \def\independenT#1#2{\mathrel{\rlap{$#1#2$}\mkern2mu{#1#2}}}

  % operators
  \DeclareMathOperator*{\argmin}{arg\,min}
  \DeclareMathOperator*{\argmax}{arg\,max}
  \DeclareMathOperator*{\supp}{supp}
  \DeclareMathOperator*{\diag}{diag}
  \DeclareMathOperator*{\Tr}{Tr}
  
  % bold vectors
  \providecommand{\0}{\mathbf{0}}
  \providecommand{\1}{\mathbf{1}}
  \renewcommand{\aa}{\mathbf{a}}
  \providecommand{\bb}{\mathbf{b}}
  \providecommand{\cc}{\mathbf{c}}
  \providecommand{\dd}{\mathbf{d}}
  \providecommand{\ee}{\mathbf{e}}
  \providecommand{\ff}{\mathbf{f}}
  \let\ggg\gg
  \renewcommand{\gg}{\mathbf{g}}
  \providecommand{\gv}{\mathbf{g}}
  \providecommand{\hh}{\mathbf{h}}
  \providecommand{\ii}{\mathbf{i}}
  \providecommand{\jj}{\mathbf{j}}
  \providecommand{\kk}{\mathbf{k}}
  \let\lll\ll
  \renewcommand{\ll}{\mathbf{l}}
  \providecommand{\mm}{\mathbf{m}}
  \providecommand{\nn}{\mathbf{n}}
  \providecommand{\oo}{\mathbf{o}}
  \providecommand{\pp}{\mathbf{p}}
  \providecommand{\qq}{\mathbf{q}}
  \providecommand{\rr}{\mathbf{r}}
  \renewcommand{\ss}{\mathbf{s}}
  \providecommand{\tt}{\mathbf{t}}
  \providecommand{\uu}{\mathbf{u}}
  \providecommand{\vv}{\mathbf{v}}
  \providecommand{\ww}{\mathbf{w}}
  \providecommand{\xx}{\mathbf{x}}
  \providecommand{\yy}{\mathbf{y}}
  \providecommand{\zz}{\mathbf{z}}
  
  % bold matrices
  \providecommand{\mA}{\mathbf{A}}
  \providecommand{\mB}{\mathbf{B}}
  \providecommand{\mC}{\mathbf{C}}
  \providecommand{\mD}{\mathbf{D}}
  \providecommand{\mE}{\mathbf{E}}
  \providecommand{\mF}{\mathbf{F}}
  \providecommand{\mG}{\mathbf{G}}
  \providecommand{\mH}{\mathbf{H}}
  \providecommand{\mI}{\mathbf{I}}
  \providecommand{\mJ}{\mathbf{J}}
  \providecommand{\mK}{\mathbf{K}}
  \providecommand{\mL}{\mathbf{L}}
  \providecommand{\mM}{\mathbf{M}}
  \providecommand{\mN}{\mathbf{N}}
  \providecommand{\mO}{\mathbf{O}}
  \providecommand{\mP}{\mathbf{P}}
  \providecommand{\mQ}{\mathbf{Q}}
  \providecommand{\mR}{\mathbf{R}}
  \providecommand{\mS}{\mathbf{S}}
  \providecommand{\mT}{\mathbf{T}}
  \providecommand{\mU}{\mathbf{U}}
  \providecommand{\mV}{\mathbf{V}}
  \providecommand{\mW}{\mathbf{W}}
  \providecommand{\mX}{\mathbf{X}}
  \providecommand{\mY}{\mathbf{Y}}
  \providecommand{\mZ}{\mathbf{Z}}
  \providecommand{\mLambda}{\mathbf{\Lambda}}
  
  % caligraphic
  \providecommand{\cA}{\mathcal{A}}
  \providecommand{\cB}{\mathcal{B}}
  \providecommand{\cC}{\mathcal{C}}
  \providecommand{\cD}{\mathcal{D}}
  \providecommand{\cE}{\mathcal{E}}
  \providecommand{\cF}{\mathcal{F}}
  \providecommand{\cG}{\mathcal{G}}
  \providecommand{\cH}{\mathcal{H}}
  \providecommand{\cI}{\mathcal{I}}
  \providecommand{\cJ}{\mathcal{J}}
  \providecommand{\cK}{\mathcal{K}}
  \providecommand{\cL}{\mathcal{L}}
  \providecommand{\cM}{\mathcal{M}}
  \providecommand{\cN}{\mathcal{N}}
  \providecommand{\cO}{\mathcal{O}}
  \providecommand{\cP}{\mathcal{P}}
  \providecommand{\cQ}{\mathcal{Q}}
  \providecommand{\cR}{\mathcal{R}}
  \providecommand{\cS}{\mathcal{S}}
  \providecommand{\cT}{\mathcal{T}}
  \providecommand{\cU}{\mathcal{U}}
  \providecommand{\cV}{\mathcal{V}}
  \providecommand{\cX}{\mathcal{X}}
  \providecommand{\cY}{\mathcal{Y}}
  \providecommand{\cW}{\mathcal{W}}
  \providecommand{\cZ}{\mathcal{Z}}

% Commenting
\RequirePackage[colorinlistoftodos,bordercolor=orange,backgroundcolor=orange!20,linecolor=orange,textsize=scriptsize]{todonotes}
\providecommand{\comment}[2]{\todo[inline,caption={}]{\textbf{#1: }#2}}%
\providecommand{\inlinecomment}[3]{%
  %\@getnewcolor%
  %\edef\@tempa{\@colstring}%
  {\color{#1}#2: #3}}%
\newcommand\commenter[2]%
{%
  \expandafter\newcommand\csname i#1\endcsname[1]{\inlinecomment{#2}{#1}{##1}}
  \expandafter\newcommand\csname #1\endcsname[1]{\comment{#1}{##1}}
}

% Use these for theorems, lemmas, proofs, etc.
\newtheorem{proposition}{Proposition}
\newtheorem{lemma}{Lemma}
\newtheorem{corollary}[lemma]{Corollary}
%\newtheorem{conjecture}[lemma]{Conjecture}
\newtheorem{definition}{Definition}
\newtheorem{remark}[lemma]{Remark}
\newtheorem{assumption}{Assumption}
\newtheorem{theorem}[lemma]{Theorem}
\newtheorem{example}[lemma]{Example}

\newtheorem{claim}{Claim}
\newtheorem{fact}{Fact}

\definecolor{mydarkblue}{rgb}{0,0.08,0.45}
\usepackage[colorlinks=true,linkcolor=blue]{hyperref} 
\hypersetup{ %
    colorlinks=true,
    linkcolor=mydarkblue,
    citecolor=mydarkblue,
    filecolor=mydarkblue,
    urlcolor=mydarkblue
}
\usepackage[capitalize,noabbrev]{cleveref}


\usepackage{url}
\def\UrlBreaks{\do\/\do-}

\usepackage[round]{natbib}
\renewcommand{\cite}[1]{\citep{#1}}



\makeatletter
\newcommand{\skipitems}[1]{%
  \addtocounter{\@enumctr}{#1}%
}
\makeatother

\begin{document}
    \homework{COM-514: Mathematical Foundations of Signal Processing}{Fall 2019}{2}{29\textsuperscript{th} November 2019}{Oriol Barbany Mayor}
    
    \problem{Quick Review of Chapter 2}
    \begin{enumerate}[label=(\roman*)]
        \item
        \begin{itemize}
            \item $A(x):=x * h_A, \quad x, h_A\in \ell^2$. It's easy to check that $A$ is a linear operator. Let $y\in \ell^2$ and for the rest of the exercise let $\alpha, \beta \in \mathbb{C}$.
            \begin{align}
                (A(\alpha x + \beta y))_n&:=((\alpha x + \beta y) * h_A)_n := \sum_{k\in \mathbb{Z}} (\alpha x[k] + \beta y[k]) h_A[n-k]\\
                &=\alpha \sum_{k\in \mathbb{Z}} x[k]h_A[n-k] + \beta \sum_{k\in \mathbb{Z}} y[k] h_A[n-k] =: \alpha (A(x))_n + \beta (A(y))_n
            \end{align}
            
            Moreover, $A$ is also shift invariant so it's LSI. To see this, let $x'[n]:=x[n-n_0]$. Then,
            \begin{align}
                (A(x'))_n:= x' * h_A := \sum_{k\in \mathbb{Z}} x'[n-k] h_A[k] := \sum_{k\in \mathbb{Z}} x[n-k-n_0] h_A[k] =: (A(x))_{n-n_0}
            \end{align}
            \item $B(x)(t):=x(t)+\text{sinc}(t),\quad x\in \cL^2(\mathbb{R})$ is clearly not shift invariant. For the rest of the exercise, let $x'(t):=x(t-t_0)$. In this case
            \begin{align}
                B(x')(t):=x'(t)+\text{sinc}(t):=x(t-t_0)+\text{sinc}(t)\neq x(t-t_0) + \text{sinc}(t-t_0)=: B(x)(t-t_0)
            \end{align}
            hence $B$ is not LSI.
            
            \item $C(x)(t):=x(2t),\quad x \in \cL^2(\mathbb{R})$. Note that
            \begin{align}
                C(x')(t) := x'(2t) := x(2t-t_0)\neq x(2(t-t_0)) =: C(x)(t-t_0)
            \end{align}
            hence $C$ is not LSI.
            
            \item $D(x):=\frac{dx}{dt},\quad x\in \cC^\infty$. It's well-known that the derivative is linear, but for completeness, let $y\in \cC^\infty$
            \begin{align}
                D(\alpha x + \beta y) := \frac{d}{dt}(\alpha x + \beta y) = \alpha \frac{dx}{dt} + \beta \frac{dy}{dt} =: \alpha D(x) + \beta D(y)
            \end{align}
            so indeed linearity holds for $D$. By chain rule we have that
            \begin{align}
                D(x')(t):=\frac{d}{dt}x'(t) := \frac{d}{dt}x(t-t_0) =:  D(x)(t-t_0) \left[\frac{d}{dt}(t-t_0)\right] =  D(x)(t-t_0)
            \end{align}
            so $D$ is also LSI.
        \end{itemize}
        
        \skipitems{1}
        \item By definition of the adjoint operator, we have that
        \begin{align}
            \lin{C(x)(t),y(t)}_{\cL^2(\mathbb{R})} &= \int_{-\infty}^{\infty} C(x)(t)y^*(t) dt := \int_{-\infty}^{\infty} x(2t) y^*(t) dt \\
            &= \lin{x(t),C^*(y)(t)}_{\cL^2(\mathbb{R})} = \int_{-\infty}^{\infty} x(t) (C^*(y)(t))^* dt
        \end{align}
        so we can see that by letting $t':=2t$,
        \begin{align}
            \int_{-\infty}^{\infty} x(2t) y^*(t) dt = \int_{-\infty}^{\infty} x(t') y^*\left(\frac{t'}{2}\right)\frac{1}{2} dt'=: \frac{1}{2}\int_{-\infty}^{\infty} x(t') (C^*(y)(t'))^* dt'
        \end{align}
        and hence $C^*(y)(t)=\frac{1}{2}y\left(\frac{t}{2}\right)$.
        \item Let $x,y\in \cH$ and let $c\geq 0$ be such that $x(t)=0$ for $|t|\geq c$ and $d\geq 0$ such that $y(t)=0$ for $|t|\geq d$. Now by definition of the adjoint operator,
        \begin{align}
            \lin{D(x)(t),y(t)}_{\cH} &= \int_{\cH} D(x)(t)y^*(t) dt := \int_{-d }^{d} \frac{dx}{dt}(t) y^*(t) dt \\
            &= \lin{x(t),D^*(y)(t)}_{\cH} = \int_{\cH} x(t) (D^*(y)(t))^* dt
        \end{align}
        
        Using integration by parts, we get that
        \begin{align}
            \int_{-d }^{d} \frac{dx}{dt}(t) y^*(t) dt &=  x(t) y^*(t) \big\rvert_{-\min(c,d)}^{\min(c,d)} -  \int_{\cH} x(t) \frac{dy^*}{dt}(t) dt \\
            &= -  \int_{\cH} x(t) \frac{dy^*}{dt}(t) dt =: \int_{\cH} x(t) (D^*(y)(t))^* dt
        \end{align}
        where the penultimate equality follows since $\cH\subset \cL^2(\mathbb{R})$, so $y^* = y$, and $x(\pm \min(c,d))y(\pm \min(c,d))=0$ by the finite support property of $\cH$ and the definition of $c,d$. So $D^*(y)(t)=-\frac{dy}{dt}(t)$.
        
    \end{enumerate}
    
    \problem{LCMV and GSC Derivation}
    \begin{enumerate}[label=(\roman*)]
        \item First of all, note that $\argmax \norm{\xx}=\argmax \norm{\xx}^2 = \argmax \frac{1}{2}\norm{\xx}^2$. We can analytically find the local maximum of a function subject to an equality constraint as in this case by using Lagrange multipliers. Let $\yy$ be the Lagrange multiplier
        
        \begin{align}
            \cL (\xx, \yy) = \frac{1}{2}\norm{\xx}^2 + \yy^* (\bb - A \xx)
        \end{align}
        which has its maximum attained at the critical point
        \begin{align}
            \nabla_\xx \cL(\xx, \yy) = \xx - A^* \yy = 0 \Longleftrightarrow \xx = A^* \yy
        \end{align}
        
        By imposing the constraint, $A\xx= A A^* \yy= \bb$.
        \item When $M \leq N$ and $A$ is of full rank, the matrix $A A^*$ is invertible and hence
        \begin{align}
            \yy = (A A^*)^{-1} \bb \Longrightarrow \xx = A^* (A A^*)^{-1} \bb
        \end{align}
        \item Using again the same trick as before, one can use the equivalent objective function $\frac{1}{2}\hh^* R_x \hh$, which gives a Lagrangian of
        \begin{align}
             \cL (\hh, \yy) = \frac{1}{2}\hh^* R_x \hh + \yy^* (\ff - C^* \hh)
        \end{align}
        
        \begin{align}
            \nabla_\hh \cL(\hh, \yy) = R_x \hh - C \yy = 0 \Longleftrightarrow R_x \hh = C \yy
        \end{align}
        
        Note that $R_x$ is invertible since $R_x \succ 0$, so $\hh = R_x^{-1} C \yy$. Again we can find the value of the Lagrange multiplier by imposing the constraint
        \begin{align}
            C^* \hh = C^* R_x^{-1} C \yy = \ff
        \end{align}
        
        \begin{assumption}
            The matrix $C$ is full rank and $P\leq M$, meaning that $\dim (\mathcal{R}(C)) = \min(M,P)= P$.
            \label{ass:1}
        \end{assumption}
        
        Finally, we have that $C^* R_x^{-1} C$ is invertible if Assumption \ref{ass:1} holds. To proof this, I will proceed in a few steps.
        
        First of all note that $R_x^{-1} \succ 0$. This is easier to see when analysing the eigenvalue decomposition of $R_x$. Given that the covariance matrix is hermitian, it can be written as $R_x = U \Lambda U^*$, where $U$ is a unitary matrix and $\Lambda = \text{diag}(\lambda_1,\dots, \lambda_n)$ is a diagonal matrix. Given that $R_x$ is positive definite, $R_x^{-1} = U \Lambda^{-1} U^*$ is also positive definite and hence invertible since $\Lambda^{-1} = (\lambda_1^{-1},\dots, \lambda_n^{-1})$ and $\lambda_i > 0 \Longrightarrow \lambda_i^{-1} > 0$. Note that invertibility is trivial to see since $(R_x^{-1})^{-1}=R_x$.
        
        By definition of positive definiteness, we say that $R_x^{-1}\succ 0$ if $\xx R_x^{-1} \xx > 0\quad \forall \xx \in \mathbb{R}^M \setminus \{\0\}$. The projection $\tilde{\xx} := C \yy$ is non-zero $\forall \yy \in \mathbb{R}^N \setminus \{\0\}$ since the nullspace of $C$ is trivial by the rank-nullity theorem. In this case, this translates to $\dim(\cN(C))+ \dim(\cR(C))=P$, meaning that $\dim(\cN(C))=0$. Hence,
        \begin{align}
        \tilde{\xx}  := C \yy = \0 \Longleftrightarrow \yy = \0 \label{eq:1}
        \end{align}
        Now we can see that $\tilde{\xx}^* R_x^{-1} \tilde{\xx} > 0\quad \forall \tilde{\xx} \neq \0$ following from the fact that $R_x^{-1} \succ 0$. So using the latter and the equivalence of \eqref{eq:1},
        \begin{align}
        \yy^* C^* R_x^{-1} C \yy:=\tilde{\xx}^* R_x^{-1} \tilde{\xx} > 0\quad \forall \yy \in \mathbb{R}^N \setminus \{0\}
        \end{align}
        and thus $C^* R_x^{-1} C \succ 0$ and hence invertible.
        
        So we can write
        \begin{align}
            \yy = (C^* R_x^{-1} C)^{-1} \ff \Longrightarrow \hh = R_x^{-1} C (C^* R_x^{-1} C)^{-1} \ff
        \end{align}
        
        Just as a sanity check, note that by setting $R_x = I$, $A = C^*$ and $\ff = \bb$, we recover the solution found in section (ii):
        \begin{align}
            R_x^{-1} C (C^* R_x^{-1} C)^{-1} \ff = A^* (A A^*)^{-1} \bb = \xx
        \end{align}
        which can be computed if Assumption \ref{ass:1} holds.
        
        \item In this last case we have an unconstrained problem so we can use simple derivation of the objective function to find where its minimum is attained.
        \begin{align}
            f(\hh_n) := (\hh_0 - C_n \hh_n )^* R_x (\hh_0 - C_n \hh_n )
        \end{align}
        
        Note that we also need Assumption \ref{ass:1} to hold, since computing $\hh_0$ needs the matrix $C^* C$ to be invertible.
        
        Given that a covariance matrix is hermitian,
        \begin{align}
            \nabla_{\hh_n} f(\hh_n) = 2 C_n^* R_x (C_n \hh_n - \hh_0) = 0 \Longleftrightarrow C_n^* R_x C_n \hh_n =  C_n^* R_x \hh_0
        \end{align}
        
        Moreover, by previous observation and under Assumption \ref{ass:1}, $C_n^* R_x C_n$ is invertible, so
        \begin{align}
            \hh_n = (C_n^* R_x C_n)^{-1} C_n^* R_x \hh_0
        \end{align}
        as claimed.
    \end{enumerate}
\end{document}

